\documentclass[12pt]{article}

\usepackage[top=1in, bottom=1in, left=1in, right=1in]{geometry}
\usepackage{listings}
\usepackage{color}

\definecolor{dkgreen}{rgb}{0,0.6,0}
\definecolor{gray}{rgb}{0.5,0.5,0.5}
\definecolor{mauve}{rgb}{0.58,0,0.82}

\lstset{frame=tb,
  language=Java,
  aboveskip=3mm,
  belowskip=3mm,
  showstringspaces=false,
  columns=flexible,
  basicstyle={\small\ttfamily},
  numbers=none,
  numberstyle=\tiny\color{gray},
  keywordstyle=\color{blue},
  commentstyle=\color{dkgreen},
  stringstyle=\color{mauve},
  breaklines=true,
  breakatwhitespace=true,
  tabsize=3
}


\begin{document}
\title{Method Comparison}
\author{Steven Hanna}
\date{\today}
\maketitle

\section{isPrime}
The objective of the \textit{isPrime} method is to determine whether or not a specified number was a prime number.  The two methods that I will be comparing are those of \textbf{Andrew Swinston} and \textbf{Mitchell Adair}.  
\begin{lstlisting}
// PrimeFinder.java
// Andrew Swinston

public class PrimeFinder {
	public static void main(String[] args) {
		System.out.println(isPrime(5));
	}
	public static boolean isPrime(int num) {
		boolean isPrime = true;
			if(num == 1) {
				return false;
			}
			if(num % 2 == 0) {
				isPrime = false;
			}
			if(num == 2) {
				isPrime = true;
			}
			for(int i = 3; i <= (num / 2); i += 2) {
				if(num % i == 0) {
					isPrime = false;
				}
			}
		return isPrime;
	}
}

\end{lstlisting}
\newpage
\begin{lstlisting}
// FindingPrimeNumbers.java
// Mitchell Adair
import java.util.Scanner;

public class FindingPrimeNumbers {

	public static void main(String[] args) {
		//System.out.println( isPrime( 4 ) );
		Scanner input = new Scanner( System.in );
		int numOfPrimes = input.nextInt();
		printPrimes( numOfPrimes );
	}
	
	public static boolean isPrime( int num ) {
		boolean prime = true;
		for (int i = 2; i<=(num/2); i++) {
			if (num%i == 0) {
				prime = false;
			}
		}
		return prime;
	}
}

\end{lstlisting}

\subsection{Memory Management}
For general calculations of memory management: 

\begin{center}
\begin{tabular}{l c r}
Java Type & Bytes Required \\ \hline \hline
boolean & 1 \\
byte & 1 \\ \hline
char & 2 \\
short & 2 \\ \hline
int & 4 \\
float & 4 \\ \hline
long & 8 \\
double & 8\\
\end{tabular}
\end{center}

\textbf{primeFinder}: $1 (boolean)  + 4 (int) + 4 (int)  =  9(bytes)$

\textbf{FindingPrimeNumbers}: $1 (boolean) + 4 (int) + 4(int) = 9(bytes)$ 
 
 
In regards to memory management, both of these methods utilize the same amount of memory. 

\subsection{Code Readability}
\textit{FindingPrimeNumbers} maintains the clearest, and most concise code.  Although this is not a true test for efficiency and accuracy, it allows the programmer to easily make changes without stumbling through many un-needed lines. 

Upon reflection, the drawback to both programs, lies in the syntax itself.  Both programs assume that \textbf{prime} or \textbf{isPrime} respectively, to be \textbf{true}.  Although the program remains operational, exceptions can be triggered that will result in a \textbf{true} return statement, when it should have been false.  This could drastically alter the results of the program.  It is much safer to assume \textbf{false}, unless proven otherwise.

\section{printPrimes}
The objective of the \textit{printPrimes} method is to print $n$ amount of prime numbers, in a row with only $10$ values each.  Because \textbf{Andrew Swinston} did not include the \textit{printPrimes} statement in his submission, \textbf(Fatima Azfar) and \textbf{Mitchell Adair's} programs will be used. 

\begin{lstlisting}
// PrintPrimes.java
// Fatima Azfar
import java.util.Scanner.*; 

public class printPrimes {
	public static void printPrimes(int num) {
		int counter = 0;
		int lineTen = 0;
		int testPrime = 2;
		while(counter<num) {
			if(isPrime(testPrime)) {
				if(lineTen<10) {
					System.out.print("  "+testPrime);		
				}
				else {
					System.out.print("\n  "+testPrime);
					lineTen=0;
				}
				counter++;
				lineTen++;
			}			
			testPrime++;
		}
	}
		public static void main(String[]args) {
		Scanner input = new Scanner(System.in);
		System.out.println("Enter Number");
		int numberOfPrimes= input.nextInt();
		printPrimes(numberOfPrimes);
		input.close();
	}
}
\end{lstlisting}

\begin{lstlisting}
// FindingPrimeNumbers.java
// Mitchell Adair
import java.util.Scanner;

public class FindingPrimeNumbers {

	public static void main(String[] args) {
		//System.out.println( isPrime( 4 ) );
		Scanner input = new Scanner( System.in );
		int numOfPrimes = input.nextInt();
		printPrimes( numOfPrimes );
	}
	public static void printPrimes( int num ) {
		int counter = 0, primeTest = 2;
		while (counter < num) {
			if (isPrime(primeTest)) {
				if (counter != num-1) {
					System.out.print( primeTest + ", " );
					counter++;
				} else {
					System.out.print( primeTest + "." );
					counter++;
				}
			}
			if (counter%10 == 0) {
				System.out.println();
			}
			primeTest++;
		}
	}
}
\end{lstlisting}
\newpage
\subsection{Memory Management}
For general calculations of memory management: 

\begin{center}
\begin{tabular}{l c r}
Java Type & Bytes Required \\ \hline \hline
boolean & 1 \\
byte & 1 \\ \hline
char & 2 \\
short & 2 \\ \hline
int & 4 \\
float & 4 \\ \hline
long & 8 \\
double & 8\\
\end{tabular}
\end{center}

\textbf{PrintPrimes}: $4(int) + 4(int) + 4(int) + 4(int) = 16(bytes)$

\textbf{FindingPrimeNumbers} $4(int) + 4(int) + 4(int) = 12(bytes)$ 

\textit{FindingPrimeNumbers} is more efficient, only using $12 bytes$.  
\subsection{Code Readability}
Personally, I find that \textit{FindingPrimeNumbers} maintains the clearest, and most concise code.  \textit{PrintPrimes} uses an additional counter, cluttering the code, which could ultimately result in future errors.

However, when executing \textit{FindingPrimeNumbers}, a strange result is displayed.  The program has each number neatly in a row, however when it has to go to a new line, it seemingly randomly decides an amount of line spaces to execute, resulitng in a jumbled mess.  Therefore, \textit{PrintPrimes} is the superior algorithim.
\end{document}